 {\huge{\textit{Abstract}} \par}{\addtocontents{toc}{\vspace{1em}} 

Building Information Modelling (BIM) is a recent innovation in technology and process that has been introduced to the UK's architecture, engineering and construction (AEC) industry as a means to increase collaboration and productivity.
This dissertation assesses the effectiveness and implications of the industry's guidance for the adoption of BIM in the context of building services engineering by identifying discrepancies with practising engineers' collaborative experiences with BIM.
Collaboration-related issues that have been experienced by building services engineers were firstly scoped out through conversations with two AEC professionals in building services.
Thereafter, the industry's authoritative BIM guidance documents were examined and a questionnaire designed to assess the extent of practising building services engineers' compliance and issues with the guidance.
%  and identify the collaboration-related issues in their BIM adoption.
% The survey consisted  of 40 AEC professionals who are or have interacted with building services engineers
Follow-up interviews were conducted with two of the survey respondents to gain deeper insights and a more complete picture of practising engineers' collaborative experiences with BIM.
% Data about collaborative experiences with practising building services engineers was then collected through a survey that was predicated on a review of the 
% The author then designed a questionnaire for practising AEC professionals who are or have interacted with building services engineers that was predicated on the 
The study revealed five discrepancies between industry guidance and practice regarding
% These discrepancies 
% highlighted issues in 
the awareness and knowledge of BIM,
software interoperability, 
the differing skill sets of consultants and contractors,
the persistent use of proprietary file formats,
and the ambiguity of levels of definition (LODs).
The discrepancies raised questions regarding the value in streamlining the building services consultant-to-contractor handover, as is recommended by the industry.
They also suggested potentially necessary reformations of consultants' and contractors' roles in the building services industry, and of LODs for more effective communication in BIM.

%The page is kept centered vertically so can expand into the blank space above the title too\ldots
%

{\vspace{2em}
\noindent
\textit{Keywords:}
BG 6;
BIM;
building information modelling;
communication;
consultant-to-contractor handover;
information flow patterns;
interoperability;
levels of definition;
LOD;
mechanical, electrical and public health engineering;
MEP;
PAS 1192-2.
}
