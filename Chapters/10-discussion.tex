\chapter{Discussion} % Main chapter title

\label{Chapter10} % Change X to a consecutive number; for referencing this chapter elsewhere, use \ref{ChapterX}

\lhead{Chapter 10. \emph{Discussion}} % Change X to a consecutive number; this is for the header on each page - perhaps a shortened title

The collaboration-related guidance offered to the industry has been examined, and the actual communication and collaboration processes of practising building services engineers assessed.
A comparison between industry guidance and practice is conducted and the discrepancies discussed in this chapter.
% The discrepancies that have been identified between industry guidance and practice are discussed in this chapter.


%----------------------------------------------------------------------------------------
%	SECTION 1
%----------------------------------------------------------------------------------------

\section{BIM Awareness and Knowledge}

% Vast variety of BIM awareness and knowledge.
% Cf. questionnaire results + interviews.
% Clients (?) and contractors (e.g. Mark who has never handed over BIM models (as far as he knows/ from what he told me)) not all delivering BIM Level 2 projects that government wants.

The industry's implementation of BIM is the government's solution for the industry to increase its collaboration.
% As of 2016, all public sector projects should be delivered to BIM Level 2 or higher.
However, the ``Industry Practice" assessment in Chapter \ref{Chapter9} revealed a wide range of awareness and knowledge of BIM.
This was evidenced in both the survey and follow-up interviews.
On the one hand, in the survey, 43.6\% of the respondents considered themselves BIM literate, whereas 38.5\% considered themselves BIM illiterate and 18\% neither nor.
Moreover, the survey showed a range of awareness of PAS 1192-2, the industry standard on BIM process during the CAPEX phase of construction projects (see Figure \ref{pas}).
The fact that 20\% of the respondents claimed to be BIM literate without having heard of or being familiar with PAS 1192-2 is a discrepancy itself which suggests a lack of BIM knowledge.
On the other hand, BIM awareness and knowledge was strongly contrasted in the author's interviews with the two experienced contractors.
Whereas the first contractor worked regularly in a BIM environment and was advanced in their BIM knowledge, the second contractor seemed much less aware of both BIM technology and process.

The wide range of BIM awareness and knowledge found in industry practice complies with the literature, in which it was said that BIM adoption was uneven across the industry \citep{DesigningBuildingsLtd2017}.
\cite{Miettinen2014} claimed that it normally takes a company two decades for it to embed a new technology into its work practice from the time the technology is introduced to the company.
The majority of the survey respondents started working with BIM sometime since 2011 (i.e. seven years ago), the year the \citeauthor{GCS11-15} mandated BIM Level 2 (see Figure \ref{bim_yrs}).
Therefore, it might not be until the 2030s (i.e. two decades later) that BIM technology and process are fully embedded into the work practices of UK AEC companies.


%----------------------------------------------------------------------------------------
%	SECTION 2
%----------------------------------------------------------------------------------------

\section{Consultant-to-Contractor Handovers}


%----------------------------
%	SUBSECTION 1
%----------------------------

\subsection{Prevention of Smooth Handovers Due to a Lack of Interoperability}

At Stage 4a, the BG 6 recommends the following to building services engineers:
``Where there is no MEP contractor selected, the design will use generic or typical components, that may be substituted during equipment procurement" \citep[p.~54]{BG62014}.
However, in reality, these substitutions can be more complicated and time-consuming than the BG 6 makes it sound.
According to the first follow-up interviewee, current BIM technology does not support intelligent substitutions of objects.
Therefore, in order to maintain the integrity and intelligence of a BIM model, it seems that most contractors are forced to re-build the entire model from scratch.

The value of contractors re-building BIM models is debatable.
On the one hand, the re-building process is inefficient because it requires extra time and effort to design the same thing over again.
The re-work thus goes against one of the purposes of BIM, which is to streamline the design process.
For this reason, clients could argue that contractors should not get paid to essentially \emph{re-do} the consultants' work.
% Contractors doing work over again because of not being able to do intelligent substitutions in BIM model.
% One can speculate that they're not paid to do this re-work either…
% There is an element of re-work/ inefficiency here which the concept of BIM is supposedly supposed to eliminate.
On the other hand, re-building the BIM model could be regarded as a valuable exercise for contractors, especially when the MEP design is complicated.
The act of re-building forces contractors to go over every detail of the design, thus allowing them to validate and intimately familiarise themselves with the design.
This could make for a smoother transition to construction and installation by avoiding unexpected mistakes and costs.
Therefore, the re-building activity may also be financially beneficial for clients.

In contrast, the lack of interoperability may not be a problem in cases where clients do not require intact BIM models.
Instead, BIM models might simply be used for coordination purposes, whereby breakdowns of model intelligence are negligible.
However, the absences of BIM model integrity and handover to the client do not comply with BIM Level 2 project delivery.

BuildingSMART has adopted a standard that allows the sharing of incremental changes in BIM models, as opposed to sharing entire models which might easily be several megabytes (MB) large.
This standard is called the Open BIM Collaboration Format (BCF) \citep{Bosche, buildingsmart}.
According to \cite{Bosche}, the use of BCF is not yet widespread, but it is anticipated that it will soon be prevalent in the industry as part of BIM Level 3.
Perhaps a version of BCF or a similar technology exists or should be developed to allow for the updating of BIM objects from generic to specific.

% Technologies seem to have linear thinking, i.e. not designed for feedback loops, changes or re-work.
% This goes against the concept of iterations in design.
% There is a similar technology that exists already as part of IFC by buildingSMART: BCF.
% Incremental changes…



%----------------------------
%	SUBSECTION 2
%----------------------------

\subsection{Consultant Vs. Contractor Knowledge and Skill Sets}

One of the government's construction strategies is to enable early contractor involvement \citep{GCS11-15}, which is something contractors themselves seem to prefer according to the follow-up interviewees.
According to \cite{Conaghan2017}, this implies that the government wishes for the MEP consultant to work together with the MEP subcontractor and use the subcontractor's proprietary BIM objects from the outset of a project.
Such a process would streamline the subcontractor's work, who then only has to touch up the model with supports, proprietary information and commissioning data etc.
Another method that was suggested to the author is for the subcontractor to appoint the consultant to work on the BIM model throughout design and construction.

These alternative work processes raise issues about the differences in the knowledge and skill sets of consultants and contractors.
Firstly, consultants do not have the market knowledge for this \citep{Conaghan2017}.
Secondly, according to the first follow-up interviewee, the consultant would require the same level of technical knowledge as somebody installing the MEP services on site in order to virtually design and install the services in BIM.
Unfortunately, MEP consultants typically have more of a theoretical knowledge about designing building services, which is inadequate for that technical level of BIM modelling.
Perhaps, as part of implementing the government's construction strategies and improving the productivity of the AEC industry, the line between MEP consultant and contractor needs to blur.

The blurring of the line between MEP consultants and contractors could potentially have social ramifications within the industry.
The traditional procurement route can be seen as being predicated on RIBA's $ 19^{th} $ century hierarchical notions, whereby the architect assembles a team to produce the drawings, which are passed on to the contractor who gets the builders to construct the facility accordingly.
Meanwhile, the client sits back and does not get very involved.
However, it looks like some of the assumptions made in the industry about those roles may be outdated today.
Now, there is a movement in the industry, initiated by the likes of \citeauthor{Latham1994} and \citeauthor{Egan1998}, to improve communication and collaboration, put the client in the centre, and involve the contractor earlier etc.
This movement is arguably flattening the hierarchy around which the RIBA PoW stages were established.
It is possible that some MEP consultants may interpret this as them losing stature, which they may find incommodious, unfair, and/ or offensive.

% Possible ramifications on regard/ pride of graduate engineers/ CIBSE members. 
% (prejudice?)
% One could argue that the current communication/ collaboration processes we are working in are predicated/ based on a 19th century (when RIBA was founded) set of values where architect assembles the team who produces the drawings which are passed on to the contractor who get the little people to build them.
% Meanwhile, the client is not very involved.
% Government wants to implement early contractor involvement.
% According to Paddy, it is implied that sub-contractors' proprietary BIM objects are used from start.
% \hl{John Simpson?} says this could possibly work if SC appoints consultant who carries out modelling throughout design and construction.
% Martin raises problem of difference in skill set between consultants and contractors.
% Install virtually…
% Line between consultant and contractor needs to blur.



% But involve the contractor too early, he may have a lazy attitude and will not push the envelope...


%----------------------------------------------------------------------------------------
%	SECTION 3
%----------------------------------------------------------------------------------------

\section{Data Exchange Through Proprietary File Formats}

\cite{NBS2014} defines BIM Level 2 as distinguished by collaborative working by which design information is shared through common file formats such as IFC.
However, according to \cite{Laakso2012} and the first follow-up interviewee, proprietary file formats are still in use.
The interviewee reported that the maj file format by Autodesk is typically used in the industry for most ductwork fabrication facilities.
The persistent use of proprietary file formats does not fully comply with BIM Level 2.
Although the interviewee had never encountered interoperability problems when using the maj file format, such issues might arise when using different BIM software programmes or CNC machines for automatic fabrication purposes.
% However, it is possible that common file formats are not needed for certain operations, such as automatic fabrication, if proprietary formats like maj are standardised in the industry.
% and they have not yet encountered any interoperability problems with it.

\begin{comment}
The contractor finished by describing their process for sending models for automatic fabrication at Stage 5.
They only automate the manufacture of ventilation ductwork.
With this, he explains, ``if you've modelled it very accurately and to certain specifications, then you can export an maj file which can be used by ductwork manufacturing machines that will cut sheet metal and form it into rectangular ductwork."
% .maj is an Autodesk file format (proprietary).
Maj is an Autodesk (thus proprietary) file format which can be exported from the Fabrication Parts plug-in on Revit.
According to the contractor, maj files are typically used in the industry for most ductwork fabrication facilities, and they have not yet encountered any interoperability problems with it.
Once the file is exported as a maj file, there is no manual intervention;
% No manual intervention once exported in .maj format. 
it simply goes into a CNC (Computer Numerical Control) machine and the ductwork is manufactured.
\end{comment}

%----------------------------------------------------------------------------------------
%	SECTION 4
%----------------------------------------------------------------------------------------

\section{Confusion with LODs}

LODs were introduced as a tool to improve the quality of communication among BIM users about the characteristics of the graphical and non-graphical content in BIM models.
However, the LOD terms seem to have created even more confusion.
LODs get mistaken for meaning coordination, the terms themselves can be ambiguous (e.g. does level of detail mean LoD or extent of information?), and they can erroneously get associated with work stages.
\begin{comment}
\begin{itemize}
	\item People mistaking LODs for coordination
	\item Association with work stages
		\begin{itemize}
			\item No clear coherence in various industry guidance documents
			\item Same numbers in UK and similar in USA (2, 3, 4 etc.)
			\item Blanket value issue
			\item Employers' misunderstanding/ confusion
		\end{itemize}
	\item Terms themselves are ambiguous (level of detail - LoD or extent of info?)
\end{itemize}

Don't actually know whether clients typically comply with industry guidance regarding setting LODs at start of projects.
However, I've been told on more than one occasion (\cite{Quigley2017} and Jon) that there is a range in clients' and/ or contractors' understanding of LODs, and thus a large range in the quality of EIRs.

PAS 1192-2, BG 6 and the NBS article were respectively published in 2013, 2014 and 2015, whereas CIBSE DE2 and the ACE Schedule of Services MEP were respectively published in 2016 and 2017.
There appears to be a correlation between the publication dates of these documents and their guidance on whether LODs are work stage related.
This may suggest that the industry realised in 2016 that LODs should not be work stage related.
%However, this is purely speculation made without any firm evidence.
More recent publications by BSI, BSRIA and NBS might confirm this speculation.
\end{comment}


%----------------------------
%	SUBSECTION 1
%----------------------------

\subsection{Poor or Pertinent Names?}

At least 90\% of the survey respondents correctly matched the terms LoD and LoI to graphical and non-graphical content, respectively, despite the fact that only 60\% of the respondents were familiar with the terms.
These results indicate that the terms may be aptly suited to their definitions.
However, the terms LoD and LoI are ambiguous because they can be used interchangeably with the common phrases ``level of detail" and ``level of information" which may not strictly refer to either the graphical or non-graphical content of BIM models.
This is unfortunately what is done in the industry guidance documents.
Therefore, to increase clarity, perhaps more pertinent and unique names should be given to describe the amount of graphical and non-graphical content in BIM models.
% Most of survey respondents associated LoD and LoI correctly with graphical and non-graphical information, respectively.
% However, author's confusion of use of phrase ``level of detail" in reading guidance documents.
% Also, association with work stages.



%----------------------------
%	SUBSECTION 2
%----------------------------

\subsection{Project- or Stage-Related?}


There appears to be a disagreement within the industry guidance documents on whether LODs are project- or stage-related.
The work stage association is due to organisations such as the BSI, BSRIA and NBS associating LODs with work stages in PAS 1192-2, BG 6 and NBS BIM Toolkit, respectively.
The work stage association is also due to the resemblance of the LOD numbers to the numbers of the RIBA PoW and CIC Scope of Services stages.
It is therefore not surprising that clients may misunderstand the meaning of LODs and that they should define LODs as blanket values for the whole project, changing at each stage.
Moreover, the `Match the Image with the Work Stage' activity in the questionnaire showed that there was a general agreement among the respondents about the amount of graphical content that the NBS and BSRIA expect at various work stages.
% It is likely that these respondents work regularly with BIM production or designing to meet LODs. 
Despite all this, the ACE Schedule of Services MEP, CIBSE DE2 and the majority of the respondents to the open-ended question “Have you noticed a difference in levels of definition from one project to the next?” agree that LODs are project-related, not stage-related.
This dissonance resonates with something a policy and regulation consultant said in the survey: ``there is no actual definition of what the various LoDs mean, the interpretation always varies, though there are similar themes".

% There is no actual definition for LODs...
Perhaps 
% author (and clients) initially misinterpreted LODs as defined in industry guidance to be stage-related.
LODs should be interpreted as follows:
LODs are simply a set of categories of various progressions of BIM model content that one can choose from and apply to different aspects of design work.
They would be typically used to communicate how much content the model should have by the end of a work stage.
This wouldn't mean that NBS LoI 4 should apply to Stage 4 for all design aspects across all construction projects.

LODs need to be more clearly defined in the guidance documents.
Otherwise, perhaps the whole concept of LODs should be replaced by a more effective communication system for BIM.
% is confusing and should be scratched (refer to BIM manager that has alternative approach to LODs).
