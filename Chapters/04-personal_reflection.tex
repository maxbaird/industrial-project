\chapter{Personal Reflection} % Main chapter title

\label{Chapter4} % Change X to a consecutive number; for referencing this chapter elsewhere, use \ref{ChapterX}

\lhead{Chapter 4. \emph{Personal Reflection}} % Change X to a consecutive number; this is for the header on each page - perhaps a shortened title


Time management has been my greatest stumbling block…


%----------------------------------------------------------------------------------------
%	SECTION 1
%----------------------------------------------------------------------------------------

\section{Gap Year}

My interest in the built environment forged as I was exploring career paths and university degrees in high school.
Thoughts about my dad’s work in pipe insulation and happy childhood memories of assisting him in his home improvement projects are what made me initially consider a career in the built environment.
My interest grew as I pondered on the role the built environment plays in climate change, in creating communities and housing people’s everyday activities, fulfilling their needs and improving their well-being etc.
Once when I was visiting someone in a hospital, I thought about all the functions/ departments that were housed in the building and the seemingly complicated job the architect had in designing the most effective floor plan layouts for such a complex building.
I was intrigued…

I struggled in high school because of the heavy workload and the primary language of instruction, French, which I had to learn from scratch when I moved to France in 2008.
After what was a difficult, trying and stressful high school period, I decided to take a year off after I graduated in 2013, both to take a breather and to prepare myself for university.
This preparation consisted of exploring my interests/ careers in the built environment and re-taking a couple high school exams, which I performed particularly poor in, to boost my chances of acceptance at university.


%------------------------------
%	SUBSECTION 1
%------------------------------

\subsection{Architecture Placement at Hultin \& Lundquist Arkitekter in Malmö, Sweden}

During the first half of my gap year I was keen to gain work experience in architecture. I applied to all the architectural companies I could find in the Yellow Pages, with help from Ung Vision (Young Vision).
Ung Vision is a local initiative funded by Svedala council\footnote{\href{https://www.svedala.se/}{svedala.se}} with support from Arbetsförmedlingen (the Swedish Public Employment Service)\footnote{\href{https://www.arbetsformedlingen.se/Globalmeny/Other-languages/Languages/English-engelska.html}{arbetsformedlingen.se in English}} and Försäkringskassan (the Swedish Social Insurance Agency)\footnote{\href{https://www.forsakringskassan.se/privatpers/!ut/p/z1/04_Sj9CPykssy0xPLMnMz0vMAfIjo8ziTTxcnA3dnQ28LdyNTQ0cAwMMjU38jby8gg30w_Wj9KOASgxwAEcD_YLsbEUAFUIRCA!!/dz/d5/L0lDUmlTUSEhL3dHa0FKRnNBLzROV3FpQSEhL2Vu/?keepNavState=true}{forsakringskassan.se in English}} that aims to help unemployed people between the ages of 16 and 24 find work \citep{ungvision:online}.
I eventually managed to get myself a ten-week paid placement at Hultin \& Lundquist Arkitekter, a small architectural practice which at the time consisted of two architects.

Working with buildings was very exciting for me; I would walk proudly with my clipboard, floor plan drawings and ruler to the 1960s apartment building the architects were going to refurbish.
I learned some of the rules used in architectural drawings and also gained some 2D modelling skills in  AutoCAD.
One of the things that still stands out to me today was how the building layout responded to the needs of the occupants in the 1960s.
Many flats had two entrances, which puzzled me until I was told that the larger door was the main entrance and the smaller a discreet `back' entrance for the servants.
Such nuggets of information/ history that explained the purpose behind the design was the beginning of my learning how design should be purposeful.


%------------------------------
%	SUBSECTION 2
%------------------------------

\subsection{Urban Planning and Civil Engineering}

Urban planning and civil engineering were two other disciplines that interested me.
My architecture placement employer put me in contact with his friend who was an urban planner.
I spent half a day in Lund's City Planning Office where I was introduced to a variety of employees and their work.

I also managed to organise a shadowing experience on a construction site in Malmö with a civil engineer from Skanska.
I asked about certain holes in the concrete slabs and was told that they were intended for the future services (pipes and cabling etc.).
I had no idea that building services were planned so well in advance and was intrigued by this added layer of complexity to my then limited understanding of the design and construction process.



%------------------------------
%	SUBSECTION 3
%------------------------------

\subsection{Exam Resits and University Search}

During the second half of my gap year I concentrated on my meeting the conditions of my university applications and studying towards my resit exams.
I got myself a tutor, recovered my old textbooks and went to the library.
I also travelled to England to visit some of the universities I had received conditional offers from; this was my first solo trip.

Unfortunately, despite all the effort, I performed worse on my exam resits than I had done on my original exams.


%------------------------------
%	SUBSECTION 4
%------------------------------

\subsection{Gap Year Conclusion/ Resolution/ Lessons}

Setting up the objectives of exploring careers and improving my exam grades towards my overall aim of getting into a university and programme that I would find interesting and enjoyable helped keep me productive throughout my gap year.
My gap year experiences helped increase my independence and initiative-taking.
It also made me more outgoing because of all the new contacts I had gone out of my way to make in order to gain the work-related experiences and meet people at the universities etc.
It may also have improved my sense of discipline and time management skills.
Thanks to my experience at the architecture firm, I gained some useful technical knowledge in drawing and CAD.

After my exploration of built environment careers I found that different elements of architecture, urban planning and civil engineering attractive and unattractive.
I was more interested in the engineering, technical, functional aspects of design, as opposed to the artistic aspect, and the strategic planning aspects to design appealed to my organisation side.
I decided that I wanted to focus on buildings, rather than other kinds of infrastructure, and to start a programme that put people at the heart of design and emphasised on sustainability to mitigate the effects of climate change.


%----------------------------------------------------------------------------------------
%	SECTION 2
%----------------------------------------------------------------------------------------

\section{First Year}

I expected Architectural Engineering to be a programme that combined Architecture and Structural Engineering with a focus on buildings as opposed to other types of infrastructure.
Therefore, I was surprised to find it was mainly about building services, the operation and internal environment of buildings.
Despite this, I felt fortunate to have `stumbled' upon this programme because I believed it would empower me to promote the often forgotten/ hidden/ non-obvious aspects of building design and construction that are crucial to the comfort and well-being of occupants as well as the efficient and sustainable operation of buildings to mitigate and adapt to the effects of climate change.

Because of my fascination with the built environment, I thoroughly enjoyed \textbf{History of the Built Environment}.
It lay the foundation for my vocabulary and understanding of the evolution of architecture.
Despite the frustrations I experienced in the required drawing exercises, they taught me some fundamental skills in drawing neatly and with proportions and perspective.
This course opened my eyes and gave me a greater appreciation for architecture.
I think it is a great shame that the course has been discontinued.

\textbf{Introduction to Design} and \textbf{Construction Technology 1} were both very useful courses.
They taught me the fundamentals to understand, describe and explain the anatomy of a building (e.g. its structure, spaces and services) and the different reasons and approaches that might be used in a building's design.
They provided several opportunities to practise and refine my drawing skills.
The courses also emphasised that design serves a purpose/ or fixes a problem and introduced me to the concept that the setting/ environment of a building should be taken into great consideration in its design and construction.

This concept of designing with the environment in mind was further emphasised in \textbf{Introduction to the Environment}, another course I found very interesting.
It taught me the effects the built environment has on the environment and introduced me to the idea of adapting building design to deal with the effects of climate change that we cannot mitigate.

\textbf{Mathematics for Engineers and Scientists 1 and 2} were useful in providing a forum to practise my mathematical skills, which is an important tool for engineers.
It did, however, cover several advanced topics, e.g. integration and matrices, I have not found any use for during the rest of my programme or any of my work experience so far.

I would consider \textbf{Building Services Technology} the cornerstone of the Architectural Engineering programme.
It provided me with a good introduction and overview of the various building services, the design and construction of which is often the primary focus of architectural engineers.

\textbf{Mechanics B} was a challenging yet enjoyable course for me.
Many AE students, myself included, found many difficulties/ hurdles in the fundamentals of this course, which I felt could have been overcome if we had taken the pre-requisite course Mechanics A like our peers in Civil Engineering had taken.
It did not help that the course leader lectured at lightning speed, brushing over important definitions and methods that we had not yet grasped.
As student representative, though, I was able to resolve the lecturer problem to some degree by addressing these problems to him.
I believe doing this improved my leadership skills and helped to start to practise my professionalism.
I say this because looking back at the email I sent the professor, I can now see that it was long-winded and not as succinct and tactful (?) a it should have been.
Despite these flaws, the issue was addressed and the situation improved.
Although we may not have specifically applied the knowledge gained from this course in our future AE related courses, I believe Mechanics B did give us an appreciation for how floating and submerged objects behave, which may be useful in our understanding of designing water and heating systems in buildings.

Overall thoughts of first year: start vs. finish.
Excited to learn about buildings.
Gap year helped erode my shyness and make me more active (asking and answering questions) in lectures etc.
My average in Semester 1 was 67.75 (a B), which in my opinion is mediocre.
I think this was due to a lack of time management, something I have always struggled with, due to my eagerness to explore all there is to university (societies, sports and trips etc.).
I believe my mediocre performance and late submissions are what may have pushed me to improve my performance in Semester 2, where my overall average was 81.25 (an A).


%----------------------------------------------------------------------------------------
%	SECTION 3
%----------------------------------------------------------------------------------------

\section{Second Year}

% OVERALL REFLECTION/ PROGRESS %

At the start of second year, I was aware of my time management problems from the previous year.
That is why I decided to drop some of my extra-curricular/ voluntary activities, e.g. being a student representative and on the the committee of the Christian Union.
Still, I did not cope very well.
This is evidenced by my failure of the course Hydraulics \& Hydrology A where I got an overall E.
This was largely due to a misunderstanding causing me the late submission of an assignment, and subsequently zero marks, and to my poor consolidation/ revision of the course causing me to fail the exam.
This gave me a scare and a wake-up call to address my time management problems more seriously.
This meant that I had to be less of a perfectionist and start being comfortable with submitting assignments that were below par.

\textbf{Design Project A} taught me more about the passive design of buildings and adaptive thermal comfort, aspects of design I believe need to emphasised more.
We were asked to digitally model our buildings; I felt disadvantaged because I did not feel confident in any CAD software, despite having had a little experience at my architecture placement.
I ended up creating some drawings by hand
I do not think the one class used to introduce us to CAD was very helpful.
Still to this day I do not feel confident in my CAD skills.
With Building Information Modelling (BIM) becoming the future of construction, I believe it is important that universities offer all construction students proper training in digital modelling and the BIM process/ mindset.
I think Heriot-Watt University has been successful in teaching the BIM process to us AE students, but the AE programme lacks training in the software.

\textbf{Acoustics and Architectural Design} was another eye-opening and enjoyable course.
It introduced me to a different world/ aspect of building design which I had not considered before.
Since then, acoustics are something I try to always consider in building design.
The course was very structured, the course leader very approachable and helpful, and the workload manageable.

\textbf{Construction Technology 2} was useful in giving me an appreciation for construction materials, their properties and functions in structures.
A highlight from this course was witnessing a steel rod stretch before it snapped.

\textbf{Energy Principles and Applications} went into depth on building services and their design.
This course did not leave much of an impression on me, although it did probably teach me a lot of the fundamentals to building services and their design.
I think my vague memories of this course may be due to a lack of colour in the course material or lack of interaction in the course.
Things do not stand out to me, so it is hard to say how this course benefited me.

\textbf{Statistics for Science} was a good course.
The knowledge I acquired here was applied to some extent in future courses, including my fourth year Dissertation.
However, the course content was dense.
I struggled to get through the last couple chapters, which negatively affected my exam performance.
Perhaps the content could be made lighter.

\textbf{Environment and Behaviour} was a really interesting course because it linked people with the built environment, the area where my interest in the built environment stemmed from.
I learned about how people perceive the environment, how the environment affects people and how to encourage environmentally responsible behaviours.
We also covered some interesting case studies and carried out Post-Occupancy Evaluations (POEs), which was useful to understand the complexity of designing a survey that easily and accurately collects building users' perceptions of a space.
I think we were meant to apply our knowledge recently gained in statistics in the analysis of the POEs, however because the two courses were occurring simultaneously, I did not feel I got a chance to absorb the statistics information before it had to be applied in Environment and Behaviour.
Perhaps Statistics for Science could have been taught at an earlier stage to allow for a better application of statistics in this course.


% OPTIONAL COURSES %

In second year we had to choose two optional courses out of a variety in the disciplines of AE, civil engineering, urban planning, computer science and foreign languages.
I already had a few foreign languages under my belt, making this option less interesting to me.
After comparing my interest in the other electives and their relevance to AE, I selected Hydraulics \& Hydrology A (a civil engineering course) and Design Project B (an AE course).

\textbf{Hydraulics \& Hydrology A} was a challenging course.
Even though I failed it, I enjoyed it.
I often enjoy courses that are heavy in mathematics and scientific reasoning because solving the problems and challenges feels rewarding.
Moreover, what I learned about the physics of fluid mechanics, the hydraulic principles underlying the design of pipeline systems and the hydrological cycle was beneficial to my understanding of the design of buildings in terms of water, heating and drainage systems, as well as the adaptation to increased rainfall/ flooding events induced by climate change.
Despite submitting an assignment past the deadline for marks, I received good feedback which would have earned me 93\% of the assignment's marks.
I think poor management of the coursework and revision in the first semester are what caused me to fail the Hydraulics \& Hydrology A exam.

\textbf{Design Project B} was a frustrating course for me.
There was very little structure to it, which in my opinion was not helpful for someone with time management problems like myself.
I think, however, it made me start to appreciate the value of going through trial and error to find a suitable design solution, that it is okay not to have the perfect solution worked out from the start (which was often my method of working).
I was dissatisfied with my final project solution and embarrassed to have it marked by my peers at the end of the course.
Fortunately (and to my surprise), I passed the course with an A.


%------------------------------
%	SUBSECTION 1
%------------------------------

\subsection{Mechanical Engineering Summer Placement at Arup in Leeds}

In the summer after second year, I was excited to have earned a mechanical engineering placement with the world renowned company, Arup, in Leeds' Buildings team.
This placement gave me my first hands-on experience in designing and sizing heating, ventilation and water systems.
I was also introduced to some of the work the electrical engineers did.
The work experience prepared me for future courses such as Electrical and Lighting Services for Buildings and the fourth year Design Project.
I was encouraged to hear about Arup's sustainable mindset to designing and felt like I could become competent and may enjoy a job as a consulting engineer in the future.

%----------------------------------------------------------------------------------------
%	SECTION 4
%----------------------------------------------------------------------------------------

\section{Third Year}

I noticed a trend in my performance in second year: I seemed to perform very well in three out of the four courses per semester, and quite poorly in the fourth course.
I still needed to improve time management skills so that I could balance out my efforts more evenly across each course.

\textbf{Design Software Applications} was a useful course, introducing us to different types of modelling programmes and their capabilities.
If it were possible, I wish we could have been given more time to familiarise ourselves with the software e.g. Integrated Environmental Solutions Virtual Environment (IES-VE) and Standard Assessment Procedure (SAP), as these prove useful in future projects and in the industry.

\textbf{Electrical and Lighting Services for Buildings} was a very challenging course, perhaps more so because the course leader was on sabbatical.
The substitute, a practising electrical engineer and Heriot-Watt alumnus, was very helpful, though maybe a bit unfamiliar with the assessment aspects of the course.
What I found extremely frustrating was the lack of background principles and the density of technical information packed into this course.
As I was not adequately prepared for the course, I struggled to understand the material and it did not help that there were some inconsistencies/ there was a lack of structure in the course notes.
It felt like almost an entire degree in electrical engineering was crammed into this 12-week course.
It also feels like this was the only course in the entire programme that attempted to teach us electrical engineering design in buildings.
Because of this, I did not feel so confident in the design of electrical and lighting systems in my fourth year Design Project, and to this day I feel weak in my understanding of electrical systems.
I strongly feel like this course should have been spread across at least two courses, through which we should have been given opportunities/ exercises to design electrical and lighting systems in buildings.
That said, I did increase my knowledge in electricity and electrical systems in buildings thanks to this course, something I have not had the opportunity to learn elsewhere.

Despite what my peers may have felt about the following course, I personally enjoyed \textbf{Procurement and Contracts}.
It helped me to understand how the construction world works: the extent of people involved in projects and the contracts that `dictate' their transactions and relationships.
Although the content may not have been directly applicable in our AE work, I think it was important to learn about these professional agreements for when we start working in the real world, to know where our place is in a professional relationship and to understand the motivations behind different procurement routes.

\textbf{Critical Architectural Studies} was an especially difficult and frustrating course, particularly with regards to the teamwork.
This course joined AE and Interior Design students.
Now, group work exercises in other courses had been manageable/ tolerable, but it was especially trying in this course.
I think it may be because of the Interior Design students' lack of experience in group work and presentations.
A few members on my team were often absent and did not do the tasks they has agreed to do.
Due to the heavy workload, this created tension/ friction within the group and stress on myself and a couple other members who were trying to pick up others' slack.
The tension escalated to a point where I asked the course leader to mediate the situation.
From this group work exercise, I learned that one cannot always be `fair' in group situations and that it may be easier/ more practical to allocate tasks according to skills/ capability and willingness to work.
I also learned from my mistake of not including the entire group in a group decision to always try and keep communication open/ transparent in groups.
Despite these difficulties, I was glad to be able to apply my knowledge in designing with consideration to the site and climate.
This contributed to my group earning the First Prize in Sustainable Design.

I really enjoyed the second semester of third year.
\textbf{Energy and Buildings} was a really interesting course that covered topics such as renewable technologies, on-site energy generation and demand side management (DSM).
The course challenged my perception/ understanding of energy efficiency and its benefits and helped me understand why it is important to reduce our energy use on a national level as well as the factors and consequences we need to consider when implementing renewable technologies.
It also touched on human behaviour with regards to energy consumption, which I found very interesting and relevant.
The course leader's rigorous marking helped me clarify my expression in reports and refine/ improve the reasons to support my decision-making in selecting certain technologies for a site.

I deepened my understanding of psychrometrics and the operation and design of air conditioners in \textbf{Thermal Performance Studies}.
These concepts are immensely useful to understand and appreciate in the AE profession.
It also required me to revisit physical principles and practise my mathematical skills, which are important knowledge and competences for an engineer.
I did struggle to communicate with the course leader at times.
I once inadvertently offended them in an email and felt like frustrated when they refused to hear my questions in class.
This experience with the lecturer taught me a few things: to be more cautious and succinct in my written communication in order to show respect and actually receive useful/ helpful responses, to take more initiative to solve problems on my own or with my peers, and to be more assertive when my last or only avenue to get an answer to my question is to ask the course leader.

In the first half of \textbf{Design Issues}, I learned about the all-important issue of health and safety (H\&S) in construction, one of the most physically dangerous professions in the world.
I learned that not only do contractors have the responsibility to ensure the H\&S of on-site workers, but that designers also share this responsibility to avoid and mitigate hazards to H\&S in the designs of buildings.
In the second half, I researched the climates and vernacular architecture of countries vastly different to what I know and have experienced.
A couple especially useful exercises were to come up with realistic solutions to make buildings more resilient to their climates, which may have changed or become more extreme due to climate change.
These were useful in terms of designing more climate-sensitive buildings, a skill that is important architectural engineers in this time of climate change.

%when designing buildings, it is important to remember the end use of the building and that the building is expected to operate for upwards of fifty years.
\textbf{Facilities Management Principles} helped me visualise the end goal and continuous aspect of the work of an architectural engineer, which is to design and \emph{maintain} buildings and systems to meet the needs of the end users, in the present and \emph{future}.
The course covered the importance of comprehensive briefs, POEs and soft landings to design functional buildings, improve the operation of existing buildings and the design of future buildings, and inform and empower the end users in the operation of their building.
It also emphasised the importance of involving facilities managers and end users in the decision-making process in construction.
This course rounded off nicely the end goal of the AE profession/ Without this course, I may not have gained a thorough appreciation for the ultimate purpose of architectural engineers: to design internal environments that meet users' needs and offer them well-being.


%------------------------------
%	SUBSECTION 1
%------------------------------

\subsection{Architecture Placement at Hultin \& Lundquist Arkitekter in Malmö, Sweden}

Tasks I did:
\begin{itemize}
	\item Produced window and door schedules on AutoCAD.
	\item Responsible for daylight calculations of a new-build during planning stage.
	\item Began a feasibility study on the construction of recycling sheds to five existing apartment buildings shortly before end of placement.
\end{itemize}

How did I progress or what did I learn during this placement?
I had to be creative in clearly displaying the numbers and types of doors, kitchens and apartments in schedules.
I also improved my conditional formatting skills in spreadsheets during my work on daylight calculations.
Practised my 2D AutoCAD skills again.
Took measurements to double-check the accuracy of outdated drawings.


%------------------------------
%	SUBSECTION 2
%------------------------------

\subsection{Electrical Engineering Summer Placement at Hoare Lea in London}

Having recently studied Electrical and Lighting Services for Buildings and already tried out mechanical engineering at Arup, I wanted to explore the other half of M\&E (mechanical and electrical) engineering.

Just preparing for my interview for the Hoare Lea placement made me take some time to reflect on skills and attributes.
It also improved on my presentation-making skills.

Unfortunately due to having not-so-engaged supervisors, I do not think I learned as much about electrical engineering as I could have.
I practised drawing an electrical schematic by hand, on AutoCAD and on Amtech.
Despite not doing much, I did learn a bit about busbars, lighting and resilient cables.

I was also asked to draw fire detection and alarm drawings, which was an exercise I learned a lot from thanks to a couple fire specialists from Honeywell.
I read related building standards to ensure my design complied with them.

Hoare Lea is a building services consultancy.
The London office in which I worked housed many different disciplines.
I therefore took the opportunity to `roam' around and explore the different types of work that I could get into after university.
I would say my experience at Hoare Lea expanded/ broadened my horizons rather than deepened/ deepening my knowledge in a specific area.
I exposed myself to/ scraped the surface of/ was introduced to eight different specialities:
electrical engineering,
fire engineering,
building optimisation,
BIM,
acoustics,
sustainability,
façade access and
vertical transportation.
I was excited to learn about the breadth of work/ career opportunities within building services.
I gained a particular interest in the building optimisation speciality where one analyses the modelled and actual performance of buildings, advises clients on how to optimise the performance of their building and reduce running costs.
Thanks to my exposure to these specialities, I have been able to discover new and exciting career paths, of which I am considering working in building optimisation.


%----------------------------------------------------------------------------------------
%	SECTION 5
%----------------------------------------------------------------------------------------

\section{Fourth Year}

Fourth year was a struggle, especially in terms of time management.
Two autonomous projects (the Design Project and Dissertation) and two 100\% coursework-based courses in the first semester…
I remember writing a report for my Acoustics Lab Project after the deadline and the discouraging feeling of pushing deadlines in other courses.

In the first semester, I left large parts of the first stage of my Design Project unfinished.
It came to a point where I was writing things I would complete over the holidays before the second stage.
I submitted several assignments late: Acoustics lab report, Thermofluids lab report, Design Project Stage 1, Dissertation interim report.

I also felt very insecure in the first semester.
I compared myself to others, believing everyone else in my class had found a way to cope with the workload.
I was ashamed to show up in DP workshops or attend 1-to-1s with the course leader in case he or other people were to see how little I had done.
I didn't dare ask questions/ for help or work alongside/ together with my peers because that again would show how far behind I was.

In the second semester it was the first time I submitted assignments before their deadlines; this was in Innovation in Construction Practice.
It may also have been due to the their smaller size in grade weighting and in my added/ particular interest in this course…
We also did not have as many deadlines to juggle in the second semester.
SIB had optional homework assignments that we could submit for feedback in preparation for the exam.
The DST and DP only had the big end-of-semester deadlines…
So there are these factors to consider which probably made semester 2 more manageable, but after falling into a hole of despair in S1, I entered S2 with a new-found motivation to improve and persevere.
I was more structured and proactive in my approach to the DP work.
I did not totally let go of my fears of inadequacy/ of being judged, but I did open myself up more to help to people I felt I could confide in e.g. the course leader and a couple friends.

The \textbf{Design Project} started with the one-week Collaborative Design Project, during which I think my team leader skills really shone.
I often take on the role of team leader in my group projects.
This time I was pleasantly surprised by the effort and contribution of my other team members.
It was a difficult week, during which I got little sleep, but somehow we pulled through having done a pretty good job that I think we could all be proud of.

I would have liked to have used my Collaborative Design Project building design for the personal Design Project.
However, that entailed more work to make the original floor plans fit into my building shape.
Knowing me and how I typically become engrossed in details and take more time than necessary to do work, I strategically decided to use the original building design.
I tried to be strategic in my time management and in my approach to my Design Project.
It turned out, however, that I spent too much time on trying to manage my time.
I also was not meeting the recommended weekly deadlines, which I postponed to do well in the other subjects where I found more pressing group work deadlines or a firmer idea of how to approach my work.

The \textbf{Laboratory Project} consisted of a couple group work formative assignments.
I found myself prioritising these rather than on my individual summative assignments/ personal endeavours, e.g. the Design Project and Dissertation, in order to not `fail'/ to be `loyal' to my teammates.
I learned that I needed to strike a balance between my efforts in group assignments and personal projects, but to ultimately do the best thing in my interest.
I was also frustrated with a partner one of the course leaders set me up with.
I found the person unreliable and stressed/ overly anxious about our work which in turn stressed and frustrated me.
I felt like I needed to do extra work to make up for my partner's absences, unpunctuality and stress-induced lack of work/ effort.
Anything to be said about my progress from this experience???

\textbf{Inclusive and Safe Environments} consisted of two assignments: a group presentation and a poster-report, which I did with a partner.
In the group presentation, I felt like I did my part, which was research and creating my section of the PowerPoint presentation.
I was also kind of leading and organising the group, setting up meetings etc.
The course leader had encouraged students who had little experience in presenting to do the presenting for this presentation.
Our group went on board this idea.
However, the presenters did not speak confidently and did not know the material, which I think may have negatively affected our grade.
It seemed like they did not do their part in preparing for the presentation, which us researchers had given them enough time for.
This reinforces a lesson I had learned after the project in Critical Architectural Studies: it may be more rewarding/ productive to allocate tasks according to strengths/ capabilities and willingness rather than based on fairness.

\textbf{Sustainable and Intelligent Buildings} challenged my time management and study skills.
There were optional formative assignments that we could do in preparation for the final exam.
Despite my wish/ efforts to regularly attempt these exercises, I completed less than half of them.
This is again due to my will to be thorough in my reading/ study/ preparation for assignments and exams, which was difficult because of the large amount of reading material.
I need to learn to read more efficiently, which involves developing my skimming skills and reading selectively.
My inability to go through all the material on this course and to test my knowledge through the formative assessments is probably the reason I did not do as well as I would have liked on the course (I got a B).

I loved \textbf{Innovation in Construction Practice}.
The topics were current, forward-looking and highly interesting to me, the course was very structured and the material concise thus manageable to get through.
Thanks to my dissertation, I was already familiar with some of the BIM content, which gave me an upper hand.
Assessment was 90\% exam and 10\% coursework.
The coursework consisted of three short reports.
It was on this course that I submitted assignments before their deadlines for the first time, even though it was probably influenced by my great interest in the course and the small size/ importance of the coursework assignments.
I managed to do this because I set aside and dedicated a more or less limited amount of time to do the assignments with the aim/ intention of completing them so that I could move on to other tasks/ coursework.
Despite these contributing factors, this was a great triumph for me as I have struggled with time management for a long time.
It gave me a little faith that I can meet deadlines, something I hope to achieve/ prove/ demonstrate/ keep up/ improve on in fifth year and in my future professional life/ career.

During my work on my \textbf{Dissertation}, I think I succeeded in working on it regularly, something I had aimed to do in other courses.
This was facilitated by my interest in my chosen subject.
However, I found myself rushing to finish writing the interim report and the dissertation itself.
So something I have learned (again) from this is that I need to be more productive by putting more effort in the final product instead of writing notes or reading etc.
This is something I should have learned by now…
I have been doing this since high school, probably earlier.
That is what I am trying to do now as I write my Personal Reflection during the first and second weeks of the first semester in year five.
While working on my Dissertation, I also learned that I need to stop trying to write perfectly from the first try.
Having the mindset that I am writing a crappy draft facilitates the writing process, giving me content to go back to edit and refine to something more polished.
I even found that writing what I thought of as a draft may actually not need changing/ tweaking for the final product.


%----------------------------------------------------------------------------------------
%	SECTION ?
%----------------------------------------------------------------------------------------

\section{General Reflections}

At one point, I thought \underline{I could never expect myself to meet a deadline, that late hand-ins were a given for me}.
While my peers worried about meeting deadlines, I stopped worrying because I believed I could never make it anyway.
I would always submit late and get a 10\% deduction.
I still aimed to score high on my assignments, but there was a slightly bigger pressure to do extra well because I expected a 10\% deduction.
Now I have some faith that I can meet deadlines thanks to my new husband who has encouraged to start as early as possible on assignments and be more proactive in my work, rather than reading around, ``preparing" myself for the assignment or test and organising my schedule and workspace, as I tend to do.
Now, those things are not particularly bad things to do, but because of the extreme amount of time I spent doing them because I would try to go into as much detail as possible to understand and do the best I can (being the perfectionist that I am), these eventually turned into forms of \underline{procrastination}.

Something I have realised about myself:
ever since I was little, \underline{I have always enjoyed helping people}, whether it was consoling someone crying in the playground or helping a friend out with their homework.
Even when I was exploring careers and university programmes in high school, I was most interested in the jobs that involved helping people or improving people's lives.
(Was I \textit{most} interested in this because I was also interested in sustainability and renewable energy etc.)
Despite this, I have always found it difficult to ask for help.
I guess this is because I thought of myself in a higher position, trying to help others, thinking I should have the strength and capacity to do things on my own.
I think I was also influenced by my parents, especially my dad, who never asks/ does not like to ask for help.
Through helping others, I should have realised that everyone needs help from time to time, and that there is no shame in that.

Throughout my years at Heriot-Watt University, I have learned that I need to \underline{balance my life} in order to achieve a sense of well-being and do well in university.
I felt sluggish and unproductive when I did not exercise; therefore I tried to start exercising at least a couple times a week during my fourth year.
This made me feel better.
Something I actually learned back in high school was to stop sacrificing my sleep to do work and to start prioritising it instead.
Now something I am trying to (re-)introduce to my life is practising mindfulness.

General reflections on my \underline{group work} experiences?
I.e. the types of people I have worked with and some strategies to cope/ handle those group situations.




%------------------------------
%	SUBSECTION 1
%------------------------------

\subsection{Summer Placement at Sunamp in Edinburgh [Industrial Project]}

Unable to express myself/ be heard

My own performance
Feel like I could/ should have done more
The limited availability of certain people (engineer Sandy) also limited what I could do.

Tried to be professional in my interactions with others, to set up meetings and write coherent emails to get my queries answered.



%------------------------------
%	SUBSECTION 2
%------------------------------

\subsection{Environmental Building Certification Summer Placement at Sweco in Malmö, Sweden}



%----------------------------------------------------------------------------------------
%	SECTION 6
%----------------------------------------------------------------------------------------

\section{Fifth Year}



%----------------------------------------------------------------------------------------
%	SECTION 7
%----------------------------------------------------------------------------------------

\section{Conclusion}

Overall I find that there is noticeable lack of content and emphasis on the electrical design of buildings.
Considering that building services engineers generally work in either mechanical or electrical engineering, I feel like AE students at Heriot-Watt University are missing out on half of the career opportunities.