\chapter{Personal Reflection} % Main chapter title

\label{Chapter4} % Change X to a consecutive number; for referencing this chapter elsewhere, use \ref{ChapterX}

\lhead{Chapter 4. \emph{Personal Reflection}} % Change X to a consecutive number; this is for the header on each page - perhaps a shortened title

%----------------------------------------------------------------------------------------
%	SECTION 1
%----------------------------------------------------------------------------------------

\section{Gap Year}

My interest in the built environment forged as I was exploring career paths and university degrees in high school.
Thoughts about my dad’s work in pipe insulation and happy childhood memories of assisting him in his home improvement projects are what made me initially consider a career in the built environment.
My interest grew as I pondered on the role the built environment plays in climate change, in creating communities and housing people’s everyday activities, fulfilling their needs and improving their well-being etc.
Once when I was visiting someone in a hospital, I thought about all the functions/ departments that were housed in the building and the seemingly complicated job the architect had in designing the most effective floor plan layouts for such a complex building.
I was intrigued…

I struggled in high school because of the heavy workload and the primary language of instruction, French, which I had to learn from scratch when I moved to France in 2008.
After what was a difficult, trying and stressful high school period, I decided to take a year off after I graduated in 2013, both to take a breather and to prepare myself for university.
This preparation consisted of exploring my interests/ careers in the built environment and re-taking a couple high school exams, which I performed particularly poor in, to boost my chances of acceptance at university.


%------------------------------
%	SUBSECTION 1
%------------------------------

\subsection{Architecture Placement at Hultin \& Lundquist Arkitekter in Malmö, Sweden}

During the first half of my gap year I was keen to gain work experience in architecture. I applied to all the architectural companies I could find in the Yellow Pages, with help from Ung Vision (Young Vision).
Ung Vision is a local initiative funded by Svedala council\footnote{\href{https://www.svedala.se/}{svedala.se}} with support from Arbetsförmedlingen (the Swedish Public Employment Service)\footnote{\href{https://www.arbetsformedlingen.se/Globalmeny/Other-languages/Languages/English-engelska.html}{arbetsformedlingen.se in English}} and Försäkringskassan (the Swedish Social Insurance Agency)\footnote{\href{https://www.forsakringskassan.se/privatpers/!ut/p/z1/04_Sj9CPykssy0xPLMnMz0vMAfIjo8ziTTxcnA3dnQ28LdyNTQ0cAwMMjU38jby8gg30w_Wj9KOASgxwAEcD_YLsbEUAFUIRCA!!/dz/d5/L0lDUmlTUSEhL3dHa0FKRnNBLzROV3FpQSEhL2Vu/?keepNavState=true}{forsakringskassan.se in English}} that aims to help unemployed people between the ages of 16 and 24 find work \citep{ungvision:online}.
I eventually managed to get myself a ten-week paid placement at Hultin \& Lundquist Arkitekter, a small architectural practice which at the time consisted of two architects.

Working with buildings was very exciting for me; I would walk proudly with my clipboard, floor plan drawings and ruler to the 1960s apartment building the architects were going to refurbish.
I learned some of the rules used in architectural drawings and also gained some 2D modelling skills in  AutoCAD.
One of the things that still stands out to me today was how the building layout responded to the needs of the occupants in the 1960s.
Many flats had two entrances, which puzzled me until I was told that the larger door was the main entrance and the smaller a discreet `back' entrance for the servants.
Such nuggets of information/ history that explained the purpose behind the design was the beginning of my learning how design should be purposeful.


%------------------------------
%	SUBSECTION 2
%------------------------------

\subsection{Urban Planning and Civil Engineering}

Urban planning and civil engineering were two other disciplines that interested me.
My architecture placement employer put me in contact with his friend who was an urban planner.
I spent half a day in Lund's City Planning Office where I was introduced to a variety of employees and their work.

I also managed to organise a shadowing experience on a construction site in Malmö with a civil engineer from Skanska.
I asked about certain holes in the concrete slabs and was told that they were intended for the future services (pipes and cabling etc.).
I had no idea that building services were planned so well in advance and was intrigued by this added layer of complexity to my then limited understanding of the design and construction process.



%------------------------------
%	SUBSECTION 3
%------------------------------

\subsection{Exam Resits and University Search}

During the second half of my gap year I concentrated on my meeting the conditions of my university applications and studying towards my resit exams.
I got myself a tutor, recovered my old textbooks and went to the library.
I also travelled to England to visit some of the universities I had received conditional offers from; this was my first solo trip.

Unfortunately, despite all the effort, I performed worse on my exam resits than I had done on my original exams.


%------------------------------
%	SUBSECTION 4
%------------------------------

\subsection{Gap Year Conclusion/ Resolution/ Lessons}

Setting up the objectives of exploring careers and improving my exam grades towards my overall aim of getting into a university and programme that I would find interesting and enjoyable helped keep me productive throughout my gap year.
My gap year experiences helped increase my independence and initiative-taking.
It also made me more outgoing because of all the new contacts I had gone out of my way to make in order to gain the work-related experiences and meet people at the universities etc.
It may also have improved my sense of discipline and time management skills.
Thanks to my experience at the architecture firm, I gained some useful technical knowledge in drawing and CAD.

After my exploration of built environment careers I found that different elements of architecture, urban planning and civil engineering attractive and unattractive.
I was more interested in the engineering, technical, functional aspects of design, as opposed to the artistic aspect, and the strategic planning aspects to design appealed to my organisation side.
I decided that I wanted to focus on buildings, rather than other kinds of infrastructure, and to start a programme that put people at the heart of design and emphasised on sustainability to mitigate the effects of climate change.


%----------------------------------------------------------------------------------------
%	SECTION 2
%----------------------------------------------------------------------------------------

\section{First Year}

I expected Architectural Engineering to be a programme that combined Architecture and Structural Engineering with a focus on buildings as opposed to other types of infrastructure.
Therefore, I was surprised to find it was mainly about building services, the operation and internal environment of buildings.
Despite this, I felt fortunate to have `stumbled' upon this programme because I believed it would empower me to promote the often forgotten/ hidden/ non-obvious aspects of building design and construction that are crucial to the comfort and well-being of occupants as well as the efficient and sustainable operation of buildings to mitigate and adapt to the effects of climate change.

Because of my fascination with the built environment, I thoroughly enjoyed \textbf{History of the Built Environment}.
It lay the foundation for my vocabulary and understanding of the evolution of architecture.
Despite the frustrations I experienced in the required drawing exercises, they taught me some fundamental skills in drawing neatly and with proportions and perspective.
This course opened my eyes and gave me a greater appreciation for architecture.
I think it is a great shame that the course has been discontinued.

\textbf{Introduction to Design} and \textbf{Construction Technology 1} were both very useful courses.
They taught me the fundamentals to understand, describe and explain the anatomy of a building (e.g. its structure, spaces and services) and the different reasons and approaches that might be used in a building's design.
They provided several opportunities to practise and refine my drawing skills.
The courses also emphasised that design serves a purpose/ or fixes a problem and introduced me to the concept that the setting/ environment of a building should be taken into great consideration in its design and construction.

This concept of designing with the environment in mind was further emphasised in \textbf{Introduction to the Environment}, another course I found very interesting.
It taught me the effects the built environment has on the environment and introduced me to the idea of adapting building design to deal with the effects of climate change that we cannot mitigate.

\textbf{Mathematics for Engineers and Scientists 1 and 2} were useful in providing a forum to practise my mathematical skills, which is an important tool for engineers.
It did, however, cover several advanced topics, e.g. integration and matrices, I have not found any use for during the rest of my programme or any of my work experience so far.

I would consider \textbf{Building Services Technology} the cornerstone of the Architectural Engineering programme.
It provided me with a good introduction and overview of the various building services, the design and construction of which is often the primary focus of architectural engineers.

\textbf{Mechanics B} was a challenging yet enjoyable course for me.
Many AE students, myself included, found many difficulties/ hurdles in the fundamentals of this course, which I felt could have been overcome if we had taken the pre-requisite course Mechanics A like our peers in Civil Engineering had taken.
It did not help that the course leader lectured at lightning speed, brushing over important definitions and methods that we had not yet grasped.
As student representative, though, I was able to resolve the lecturer problem to some degree by addressing these problems to him.
I believe doing this improved my leadership skills and helped to start to practise my professionalism.
I say this because looking back at the email I sent the professor, I can now see that it was long-winded and not as succinct and tactful (?) a it should have been.
Despite these flaws, the issue was addressed and the situation improved.
Although we may not have specifically applied the knowledge gained from this course in our future AE related courses, I believe Mechanics B did give us an appreciation for how floating and submerged objects behave, which may be useful in our understanding of designing water and heating systems in buildings.

Overall thoughts of first year: start vs. finish.
Excited to learn about buildings.
Gap year helped erode my shyness and make me more active (asking and answering questions) in lectures etc.
My average in Semester 1 was 67.75 (a B), which in my opinion is mediocre.
I think this was due to a lack of time management, something I have always struggled with, due to my eagerness to explore all there is to university (societies, sports, trips and events etc.).
I believe my mediocre performance and late submissions are what may have pushed me to improve my performance in Semester 2, where my overall average was 81.25 (an A).


%----------------------------------------------------------------------------------------
%	SECTION 3
%----------------------------------------------------------------------------------------

\section{Second Year}



%------------------------------
%	SUBSECTION 1
%------------------------------

\subsection{Mechanical Engineering Summer Placement at Arup in Leeds}


%----------------------------------------------------------------------------------------
%	SECTION 4
%----------------------------------------------------------------------------------------

\section{Third Year}


%------------------------------
%	SUBSECTION 1
%------------------------------

\subsection{Architecture Placement at Hultin \& Lundquist Arkitekter in Malmö, Sweden}



%------------------------------
%	SUBSECTION 2
%------------------------------

\subsection{Electrical Engineering Summer Placement at Hoare Lea in London}



%----------------------------------------------------------------------------------------
%	SECTION 5
%----------------------------------------------------------------------------------------

\section{Fourth Year}


%------------------------------
%	SUBSECTION 1
%------------------------------

\subsection{Summer Placement at Sunamp in Edinburgh [Industrial Project]}


%------------------------------
%	SUBSECTION 2
%------------------------------

\subsection{Environmental Building Certification Summer Placement at Sweco in Malmö, Sweden}



%----------------------------------------------------------------------------------------
%	SECTION 6
%----------------------------------------------------------------------------------------

\section{Fifth Year}



%----------------------------------------------------------------------------------------
%	SECTION 7
%----------------------------------------------------------------------------------------

\section{Conclusion}

Overall I find that there is noticeable lack of content and emphasis on the electrical design of buildings.
Considering that building services engineers generally work in either mechanical or electrical engineering, I feel like AE students at Heriot-Watt University are missing out on half of the career opportunities.