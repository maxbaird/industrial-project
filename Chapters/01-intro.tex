\chapter{Introduction} % Main chapter title

\label{Chapter1} % Change X to a consecutive number; for referencing this chapter elsewhere, use \ref{ChapterX}

\lhead{Chapter 1. \emph{Introduction}} % Change X to a consecutive number; this is for the header on each page - perhaps a shortened title

Building Information Modelling (BIM) is a recent innovation in technology and process that has been officially introduced to the UK's architecture, engineering and construction (AEC) industry as a means to increase collaboration and thereby improve productivity.
In the past ten years, protocols, guides and standards have been published to support the industry's adoption of BIM in its work processes.
Building services engineering, a.k.a. mechanical, electrical and public health (MEP) engineering, is one of the core disciplines of the AEC industry.
The design, manufacture and installation of building services are complex, demanding the use of various software and the collaboration of multiple parties, such as consultants, contractors and specialist designers.
% Building services are a complex aspect of buildings
% Additionally, the market for building services plant and equipment is one of the most vast markets for construction products.
This dissertation assesses the effectiveness and implications of the UK industry's BIM guidance in the context of building services engineering by identifying discrepancies with practising engineers' collaborative experiences with BIM.


\begin{comment}
The topic of interest of this research study is communication and collaboration in the architecture, engineering and construction (AEC) industry in the UK from the perspective of a building services engineer.
\hl{This interim report} starts off with a literature review that provides some context for the need for improved communication and collaboration in the industry.
It then summarises particular collaboration-related concerns raised by building services professionals during scoping discussions.
\hl{These concerns gave shape to the aim of the research study: to investigate and compare building services engineers' ``prescribed" processes for communication and collaboration with other stakeholders during Stages 4 and 5 of RIBA Plan of Work 2013 with reality.}
The report concludes with an outline of the objectives, methodology and schedule.
\end{comment}

