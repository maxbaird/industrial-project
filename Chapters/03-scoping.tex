\chapter{Scoping Discussions} % Main chapter title

\label{Chapter3} % Change X to a consecutive number; for referencing this chapter elsewhere, use \ref{ChapterX}

\lhead{Chapter 3. \emph{Scoping Discussions}} % Change X to a consecutive number; this is for the header on each page - perhaps a shortened title

In order to scope out existing collaboration-related issues that have been experienced by building services engineers, two AEC professionals in building services were approached.

Conversations with \cite{Quigley2017} and \cite{Conaghan2017} have highlighted some problems of communication and collaboration in the industry from a building services engineer's perspective.
\cite{Quigley2017} was in charge of BIM processes and production in Hoare Lea's London office.
Hoare Lea is a firm of consulting engineers specialising in mechanical, electrical, and public health (MEP) engineering.
Quigley expressed a frustration with the ambiguous nature of the information delivery process.
According to him, in the interest of increasing the efficiency of the design and construction process, there was a need to clarify what consulting MEP engineers had to deliver information-wise at each stage of the process.
He indicated that there was an issue in the Building Services Research and Information Association's (BSRIA's) guide \textit{BG 6/ 2014 – A Design Framework for Building Services} that was still not fully resolved.

\cite{Conaghan2017}, a consultant at Hoare Lea and Vice President at CIBSE, elaborated on these concerns.
A particular point of difficulty was identified in the transition from Stage 4 (Technical Design) to Stage 5 (Construction) of the RIBA PoW 2013.
According to the RIBA PoW 2013, the \textit{consulting} building services engineer's work concludes at Stage 4 and the \textit{contracting} building services engineer's work begins at Stage 5.

\cite{Conaghan2017} raised two problems that occur during the shift from Stage 4 to 5.
Firstly, after all the work the consulting building services engineers would put into building a comprehensive and intelligent BIM model, the contractors would simply abandon the model and create a new one.
One of the reasons contractors would abandon the model was because the consultant's model, though intelligent, was generic, and if the contractor were to replace the model's generic data with dimensionally exact manufacturer's product data from their own library, it would \say{break} the intelligence of the model, thus making it \say{dumb}.
This suggests an underlying problem of interoperability.
Another reason is that the contractor's own dimensionally exact model could be sent directly to workshops for automatic fabrication of pipework elements, for example; this would not be possible with the consultant's generic model.
The current mode of work is wasteful in terms of time and cost, going against one of the purposes of BIM which is to reduce inefficiencies in the information supply chain \citep{NBS2014}.

Secondly, consulting building services engineers require contractor input earlier than Stage 5.
Since the building services market is constantly evolving and consultants are not positioned to intimately know the market, it would be unreasonable to expect them to produce a BIM model complete with accurate manufacturer's product data that is ready for the contractors to implement on the construction site.
Rather, it would be more reasonable for the contractor, who does intimately know the market, to feed their expertise into the model before the end of Stage 4.
This is what BSRIA recommends in the BG 6 \citep{BG62014}.
The BG 6 is the building services engineer's `bible' that provides guidance on their activities, as well as drawing and model definitions for each stage of the design and construction process.
The BG 6 splits Stage 4 into three parts: 4a, 4b and 4c. 
Contrary to the RIBA PoW, the BG 6 suggests involving the contractor at an earlier stage, specifically at Stage 4c, to make for a smoother transition to the construction stage.
However, according to \cite{Conaghan2017}, there is an ongoing conflict in the industry about the definitions of Stages 4 and 5, and the point at which to involve the contractor.

The scoping discussions have highlighted problems in the understanding of work stage deliverables and in consultant-to-contractor handovers regarding poor interoperability, automatic fabrication and the timing of the handover.
